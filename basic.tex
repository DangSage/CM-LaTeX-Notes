\documentclass[10pt]{article}
\usepackage[utf8]{inputenc}
\usepackage[T1]{fontenc}
\usepackage{amsmath}
\usepackage{amsfonts}
\usepackage{amssymb}
\usepackage[version=4]{mhchem}
\usepackage{stmaryrd}
\usepackage{graphicx}
\usepackage[export]{adjustbox}
\graphicspath{ {./images/} }
\usepackage{hyperref}
\usepackage{geometry}
\geometry{left=1in, right=1in, top=1in, bottom=1in}

\title{A Basic Introductory to LaTeX}
\author{Ethan Khai Dang\thanks{An Undergraduate at University of Massachusetts, Lowell. Graduating in 2026.}}
\date{2024 March 21}

\begin{document}

\maketitle
\thispagestyle{empty} % Optional: Removes page number from title page
\newpage

\begin{abstract}
In general this is a really easy way we can get started with formatting and simply just getting things on paper without thinking about it.
Of course, some of this is going to be formatted strangely in ways you can't really predict quite yet.

Despite that, this should not deter you from learning the basics of what a LaTeX workflow can look like! There are some quirks,
but we can easily create some scalable documents that look really nice and are easy to read. Here are some of them.
\end{abstract}


As you may notice, every time we hit a new line, it doesn't actually start a new line in the document. This is because we need to use two backslashes to start a new line. This is a little bit strange, but it's something you'll get used to.
This is a new line.\\We are passed the newline! You can see that we are now on a new line.\\And now there's another!
\\Newline 1\\Newline 2\\Newline 3

Of course we need an actual way to compile this as well. To compile, you can use the following command:
\begin{verbatim}
    pdflatex basic.tex
\end{verbatim}
This will create a file called \texttt{basic.pdf} that you can open and view. In compiling this document, we created a bunch of auxiliary files that we don't really need to worry about.\\
With the command \texttt{latexmk -c} we can remove all of these files. This is however a different command apart of the Ubuntu package manger, using the following command
\begin{verbatim}
    sudo apt-get install latexmk
\end{verbatim}

You may notice that the formatting is a little bit strange on the pdf. This is because we haven't actually told LaTeX to start a new page yet. We can do this by using the \texttt{newpage} command. This will start a new page in the document.\\
\newpage
This is a new page!\\
\newpage
% create a new page
\section{Sections}
We can also create sections in the document. This is done using the \texttt{section} command. This will create a new section in the document.\\\\
\textbf{Subsections}\\
We can also create subsections in the document. This is done using the \texttt{subsection} command. This will create a new subsection in the document.\\\\
\textit{Subsubsections}\\
We can also create subsubsections in the document. This is done using the \texttt{subsubsection} command. This will create a new subsubsection in the document.\\\\
\textbf{Bold Text}\\
We can also create bold text in the document. This is done using the \texttt{textbf} command. This will create bold text in the document.\\\\
\textit{Italic Text}\\
We can also create italic text in the document. This is done using the \texttt{textit} command. This will create italic text in the document.\\\\
\texttt{Typewriter Text}\\
We can also create typewriter text in the document. This is done using the \texttt{texttt} command. This will create typewriter text in the document.\\\\
\textbf{Lists}\\
We can also create lists in the document. This is done using the \texttt{itemize} command. This will create a list in the document.\\
\begin{itemize}
    \item This is the first item in the list.
    \item This is the second item in the list.
    \item This is the third item in the list.
\end{itemize}
\newpage
\section{Package Management}
We can also use packages in the document. This is done using the \texttt{usepackage} command. This will allow us to use packages in the document.\\
\begin{verbatim}
    \usepackage{amsmath}
    \usepackage{amssymb}
    \usepackage{graphicx}
    \usepackage{hyperref}
\end{verbatim}
This will allow us to use the \texttt{amsmath}, \texttt{amssymb}, \texttt{graphicx}, and \texttt{hyperref} packages in the document.\\
\section{Conclusion}

In general, this is a lot to learn all at once, but it genuinely proves to be a very fast tool to use.
This kind of format allows us to not need to think about the formatting of the document as much. We can just write and let LaTeX handle the rest.\\
Compiling is not so hard, and we can easily get a nice looking document.\\
Of course other functionality is available, and that will require using packages.

\end{document}