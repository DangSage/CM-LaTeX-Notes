\documentclass{article}

\title{A Basic Introductory to LaTeX}
\author{Ethan Khai Dang\thanks{An Undergraduate at University of Massachusetts, Lowell. Graduating in 2026.}}
\date{2024 March 21}

\begin{document}

\maketitle
\thispagestyle{empty} % Optional: Removes page number from title page
\newpage

\begin{abstract}
In general this is a really easy way we can get started with formatting and simply just getting things on paper without thinking about it.
Of course, some of this is going to be formatted strangely in ways you can't really predict quite yet.

Despite that, this should not deter you from learning the basics of what a LaTeX workflow can look like! There are some quirks,
but we can easily create some scalable documents that look really nice and are easy to read. Here are some of them.
\end{abstract}


As you may notice, every time we hit a new line, it doesn't actually start a new line in the document. This is because we need to use two backslashes to start a new line. This is a little bit strange, but it's something you'll get used to.
This is a new line.\\We are passed the newline! You can see that we are now on a new line.\\And now there's another!
\\Newline 1\\Newline 2\\Newline 3

Of course we need an actual way to compile this as well. To compile, you can use the following command:
\begin{verbatim}
    pdflatex basic.tex
\end{verbatim}

This will create a file called \texttt{basic.pdf} that you can open and view. You can also use the following command to create a \texttt{.dvi} file:
\begin{verbatim}
    latex basic.tex
\end{verbatim}

\end{document}