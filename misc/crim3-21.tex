\documentclass[10pt]{article}
\usepackage[utf8]{inputenc}
\usepackage[T1]{fontenc}
\usepackage{amsmath}
\usepackage{amsfonts}
\usepackage{amssymb}
\usepackage[version=4]{mhchem}
\usepackage{stmaryrd}
\usepackage[export]{adjustbox}
\usepackage{hyperref}
\usepackage{geometry}
\geometry{left=1in, right=1in, top=1in, bottom=1in}


\title{Intro to Criminal Justice Notes}
\author{Rebecca Dunlea}

\begin{document}
\maketitle
\section*{3-21: Sentencing Lecture}
\subsection*{Sentencing}
\begin{itemize}
  \item The imposition of a criminal sanction by a judicial authority
    \item The punishment of a person convicted of a crime
        \item The process of making a judgment based on the evidence presented in court
\end{itemize}
\subsection*{Sentencing Goals}
\begin{itemize}
  \item Retribution
    \item Deterrence
        \item Incapacitation
            \item Rehabilitation
                \item Restoration

\end{itemize}

\subsection*{Sentencing Models Locally in Massachusetts}
The level of the crime and the criminal history of the defendant are the two most important factors in determining the sentence.

In the past there was even a bootcamp sentencing option for criminals. This is still used in a small number of states in the USA.

More Commonly used punishments include the following:
\begin{itemize}
    \item fines
    \item probation
    \item fees
        \item to the courts
        \item to the victim
        \item anywhere between a couple hundred to thousands of USD
    \item restitution
        \item to the victim
        \item to the state
\end{itemize}
In general judges can get creative with their sentencing options. They can also use a combination of the above options.

In one example, in Painesville, Ohio, a municipal court Judge uses a combination of fines and community service to punish people. As Judge Chiccineti said "You can take each case and tailor it to the individual. You can make it fit the crime and the person."

\section*{Indeterminate Sentencing}
\subsection*{Parole}
You will serve the sentence but in fact you will serve only a range rather than a specific length of time. After the minimum sentence a parole board determine whether the inmate is ready for reentry. This idea really comes from the idea that once you have done the minimum of a sentence, you will actually serve the minimum and become eligible for parole. They will determine your eligibility after this minimum sentence based on your circumstances and meeting with you. If the board decides so you can be released to your community. These decisions are final and can't be appealed.\\\\
Parole board decisions are explicitly used for \textbf{prison sentences} and \textit{not} for \textbf{jail sentences.}\\
\end{document}